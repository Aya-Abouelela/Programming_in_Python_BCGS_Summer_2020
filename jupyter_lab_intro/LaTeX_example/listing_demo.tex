% Short demo of the listing-package to include source code
% into LaTeX documents
\documentclass[12pt, ngerman]{scrartcl}

\usepackage[utf8]{inputenc}
\usepackage[T1]{fontenc}
\usepackage{babel}
\usepackage{csquotes}

% Zum Einbinden von Programmcode verwenden wir das listings-Paket
\usepackage{listings}

% textcomp is necessary to typeset quotation marks correctly in
% code-segments (option upquote=true in listings)
\usepackage{textcomp}

% For syntax highlighting:
\usepackage{xcolor}

% lstset sets global settings for the listings package

% The following settings allow th usage of German-Umlaute
% and the tilde in listing-environments:
\lstset{
  basicstyle=\ttfamily,
  literate={~} {$\sim$}{1} % set tilde as a literal
           {ö}{{\"o}}1
           {ä}{{\"a}}1
           {ü}{{\"u}}1
           {ß}{{\ss}}1
           {Ö}{{\"O}}1
           {Ä}{{\"A}}1
           {Ü}{{\"U}}1
}

% style for bash-shell code
\lstdefinestyle{bash}{language=bash,%
 commentstyle=\color{red},       % comment style
 keywordstyle=\color{blue},      % keyword style
 upquote=true,     % correct quotation marks in Bash-Codes
 showspaces=false, % Do not emphasize Spaces
 showstringspaces=false,
 breaklines=true,  % split long code-lines
 postbreak=\mbox{{$\hookrightarrow$}\space}
}

% define a new enviroenment for bash-codes within the
% document
\lstnewenvironment{lstbash}
{ \lstset{style=bash} } {}

\newcommand{\lstinputbash}{\lstinputlisting[style=bash]}

\begin{document}
%
\section{Codes}
Insert code - no specific proghramming language - with the
\texttt{lstlistings}-enviroenment:
%
\begin{lstlisting}
user$ ls
datei_1.txt  datei_2.txt
\end{lstlisting}
%
Now insert some \texttt{bash}-code. We defined an environment for it:
%
\begin{lstbash}
# Here a simple Bash for-loop
for NUMBER in 1 2 3 4 5
  touch datei_${NUMMER}.txt
done
echo 'Loop finished'
\end{lstbash}
%
Here another example to insert code into the text:
The command
\lstinline{mv *txt copy} moves all textfiles into the folder
\lstinline{copy}.

To finish, we insert \texttt{bash}-code from a script file. We defined for this task
the command \texttt{lstinputbash}:
%
\lstinputbash{for_loop.sh}

\end{document}
